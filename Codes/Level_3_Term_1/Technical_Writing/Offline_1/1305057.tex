\documentclass{report}


\usepackage{color}

\begin{document}

\newcommand{\stylus}[1]{\textbf{\textit{#1}}}
\newcommand{\style}[2]{\textcolor{red}{#1} \textcolor{blue}{#2}}

\title{\textcolor{red}{\LaTeX assignment on ``The Elements of Style"}}
\author{\textcolor{blue}{\textbf{Md.Saqib Hasan,1305057}}}
\date{\textcolor{green}{\textit{3rd October,2016}}}
\maketitle
\newpage





\tableofcontents



\chapter{Introduction}
This book was first written by William Strunk Jr. in 1918. He was a professor at Cornell University. The book was used as an in-house guide and was nicknamed "the little book". E.B. White was a student of Cornell then and studied under William. Later, after passing, he forgot about the little book but later reminded himself of it. He then went to add the last chapter and a new edition was published.
\\
\\
The book is a short,precise,clean and lucid handbook on the basic grammar of writing. It's clarity and size are what makes it exceptionally useful for students studying english. It has been named by Times as one of the 100 most influential books written in English since 1923.



\chapter{\stylus{Elementary Rules of Usage}}
\section{Rule of adding `s}
The rule is to use `s to show possession except in some special where it's(this here means it is) not needed.
\begin{itemize}
\item They are Charlie's angels.
\item The book of Moses. - exception
\item It's (it is) his bicycle.
\end{itemize}




\section{Rule of commas}
{
\em
Commas are one of the most important punctuations in English language. As such, there proper use is epitome to writing perfection. Most important are these rules:

\begin{enumerate}
\item When joining multiple sentences with conjunctions, use serial commas to seperate the objects. In case of business names, last comma is ommitted.
\item Descriptive sentences, that would otherwise be in parentheses, should be placed in between commas. Anything else is unacceptable. In case of dates,noun identification or restrictive clauses, there is an exception.
\item When introducing an independent clause, always use a comma unless the relationship is too close in the case of`and'.
\item Never join independent clause with commas. A semi-colon(;) is much preferred. However, exception lies in case of short and extremely similar sentences.
\end{enumerate}
Examples are as follows:
\begin{description}
\item[Rule 1] There were red,blue,green, and yellow LEGO bricks
\item[Rule 1] Bruce,Jenkins \& Lewis (No comma before the last one)
\item[Rule 2] My people, the dweller of this region, are pleased with your gifts.
\item[Rule 2] Today is 16th February,2017.
\item[Rule 2] 7 May 2018 (Special case and much preferred)
\item[Rule 2] Md.Rashif Hossain,Ph.D.,is from CS background.
\item[Rule 2] The poet Jashim Uddin (exception)
\item[Rule 2] People who cheat people should not expect honesty from others. (restrictive)
\item[Rule 3] I have heard the cry of Jesus, but am still not faithful.
\item[Rule 3] He is a skilled developer and thoroughly competent for the job.
\item[Rule 4] Duran is a great poet;he should publish his works more often.
\item[Rule 4] Live today, die tomorrow.
\end{description}
}





\section{Rule of breaking and joining of independent clauses}
Never replace instances of commas with periods. It is not appropriate
\begin{itemize}
\item I saw him today. Getting off the bus at Shantinagar. (Wrong)
\item I saw him today, getting off the bus at Shantinagar. (Right)
\end{itemize}


\section{Rule of colon to express a list of things}
Colons are a great way to express a list in a sentence. It should also be used to express quotations or special clauses in names,books and titles .
\begin{enumerate}
\item Three things can destroy any man: wealth,wine and women.
\item This winter reminds me of what Ned Stark said:"Winter is coming."
\item Mediation: The Art of Living
\end{enumerate}


\section{Rule of dash to state a break in order to summarize}
Dash is a very handy tool to be used to express a mark of separation stronger than a comma but weaker than a period. For example: His first coherent business decision-he never makes many- was to create a new ranking system.



\section{Rule of number of subjects and their verbs}
The verb should always match with the number of subjects being dealt with.
For example ``One of the boys has done the crime." is wrong whereas ``One of the boys have done the crime." is correct since there are multiple subjects here.



\section{Rule of pronoun and participle phrase}
There are many rules regarding pronouns.
\begin{enumerate}
\item Personal pronouns, as well as who, change form when they act as subject or object. For example ``The criminal,surprisingly, turned out to be him" is correct and not ``The criminal,surprisingly, turned out to be he.
\item The pronoun of a comparision is nominative if it acts as the subject of a stated or understood verb. For example, " Wendy plays better than I" and not "Wendy plays better than I play".
\item Avoid ambiguous verb uses such as ``He loves Sanders more than I" with more concrete and definite sentences such as ``He loves Sanders more than I do".
\item Use simple personal pronouns as subjects such as ``He loves meat pie."
\item Gerunds required the possessive pronoun. For example, ``The coach admired our teamwork."
\item Use pronouns carefully when diffrentiating between verbal principle and gerund. For example, the sentence ``Do you mind me inquiring?" and ``Do you mind my inquiring?" mean two different things.
\end{enumerate}
As for participle phrases they must always refer to the grammatical subject in all cases. For example:
\begin{itemize}
\item Running in wet boots, the thief was barely able to run fast.
\item Being a man of the sea, the sailor once again started his voyage to the high seas.
\end{itemize}
\newpage




\chapter{\stylus{Elementary Principles of Composition}}\label{chap:cr}


\section{Having a consistent and suitable writing design}\label{sec:sr}
{
\em
From the very start, you should choose a style of writing and stick with it. Choice of writing design is very important and,in many cases, determine how much convincing you seem to your audience. Every writing has a structural shape such as a sonnet has fourteen lines of five feet long each etc. The thoughts of the writer need not be coherent but the way it is delivered needs to be. Sometimes, the writer does not even need to have any specific shape of the writing such as when writing love letters. However, once you choose your design, you must stick through it till the end. Changing design and style midway can result in the writing to become less beautiful or clear.
}
\section{Paragraph=Composition Unit}
\paragraph{A paragraph means a structure of writing where we begin the start with a bit of a gap and then continue. When the paragraph ends, a significant gap is given before the next paragraph is started. The paragraph is the standard unit of composition. All types of writing compose of many paragraphs. In dialogue,for example, each sentence is itself a paragraph.}
\paragraph{Each writing is divided into various paragraphs. Usually,in essays which deal with a lot of subjects, each topic is dealt with in a single paragraph. A paragraph can be started with a concise and comprehensive sentence which indicates the direction of the paragraph. However, this has become too conventional in writing these days so many just opt to state the subject matter and then continue.In animated narratives, the paragraphs are short and abrupt and usually have no topic sentence. They are fast and fluid, moving through the events at quick speed}
\paragraph{Many people can visualize better when long,monotonous written blocks of writing are broken down into two or more paragraphs. Usually, large amounts of small paragraphs go easier on the eyes than vice versa.}
\section{Active Voice over Passive Voice}
Read the following two sentences:
\begin{description}
\item[Active] Mary was writing a poem about love.
\item[Passive] A poem,about love, was being written by Mary.
\end{description}
As it can be clearly seen, the second sentence represents the archaic ages of English writing. These types of sentences are familiar to older generations. However, the new generation prefer quick,strong affirmation of actions. Hence, their english should also be as such. The direct and abrupt nature of active voice is much more enticing to the modern writers than passive voice.Also active voice make the sentences much shorter and removes ambiguity. Using active voice should be on the agenda of every writer nowadays.
\section{Use of positive sentences over negation}
Avoid negative sentences as much as possible. Constant uses of `not' and `never' usually turn off the readers and make less impact. Be assertive. Write in a more positive fashion.In case of writing a negative sentence, think of a positive alternative and write that. Also, the practice of writing both in opposition is a great practice.

\section{Use of ``to the point" language}
Be decisive. Do not beat about the bush. Use exact language and concrete language. Great writers have done so.

\section{Constant mistakes in style}
Here, we discuss some constant mistakes of style.
\subsection{Useless words}
Avoid useless and unnecessary words as much as possible. Many a times, people write pointless adjectives and adverbs to seemingly make their writing more vivid. Avoid this.
\subsection{Loose Sentences}
Do not write in a continuous stream of loose sentences where each sentence has no connection or meaning with the next. They make your writing incoherent and unreadable.
\subsection{Tense}
In composition, keeping to a single tense is crucial for the reader's understandability.Changing tenses as you write will only serve to confuse the reader about the chain of events. Journalists, novelists all use a single tense when writing a report or chapter of a novel respectively.
\subsection{Co-ordination of ideas and emphasis}
Ideas, similar in nature, should be expressed in parallel construction such as "Long live the king;for without him we would cease to exist." A weak writer will constantly overlook this and change his type and style of expression among related ideas. It is also a good practice to place emphasised ideas at the end of a paragraph.




\chapter{\stylus{A Few Matters of Form}}
{\itshape
Some matters of form must be taken into account when writing. These are discussed in the following list:
}
\section{Colloquialisms}
If you plan to use slangs then use them without drawing attention to the reader. Do not use exclamation marks for simple sentences.
\section{Headings}
Leave a lot of space at the top of any publication for a heading. Usually headings are later expanded so this practice comes in handy.
\section{Hyphen}
Use hyphen to join two words into a single one. However, avoid using it in cases where the two words can be joined together directly, like `waterfowl'.
\section{Margins}
Keep the width of both the left and the right margins the same.
\section{Numerals}
Instead of spelling out numbers, use numerals or the Roman notation instead. However, in case of quoted dialogues, spelling them out is best.
\section{Parentheses}
We should punctuate sentences with parentheses in a way that they never existed. For example, "He,Harry Potter(the Boy Who Lived), is an orphan."
\section{Quotation}
A quotation written after a verb is preceded by a comma before the quotation mark. If a sentence starts with a quotation then a comma follows after the second quotation mark.
\section{References}
Write references in numeric or simple format instead of spelling them out. For example, write "third scene of first act" as " in I.iii"
\section{Syllabication and Titles}
For syllable or division, always consult a dictionary to know the proper way. Always write titles in italics and omit any sort of quotation marks.



\chapter{\stylus{Words and Expressions Commonly Misused}}
The following words are described to be commonly mistaken in writing:
\begin{description}
\item[Aggravte,irritate:]while the first means to add to already painful condition, the other means to cause the condition.
\item[Allusion:]confused with illusion,which means false image, while this means a reference.
\item[Among,between:]among is used when two or more people are dealt with while between is strictly with two persons.
\item[Character:]used,in most cases, in a redundant manner.
\item[Data:]Data is a plural term but is now largely considered singular in reference. Thus, the address verb should also be singular.
\item[Each and every one:]A jargon that should be avoided. `Everyone is a' better alternative.
\item[Nature:] used redundantly like character.
\item[Personally:] often used unnecesarily and can be avoided.
\item[Shall,will:] In formal writing, shall is for first person while will is for second person. However, in relaxed speech, they are seldom cared about.
\item[Would:]used to expressed actions which are,in nature, habit or repetitive. However, for phrases once a year,every day, would can be skipped.
\end{description}





\chapter{\stylus{An Approach to Style}}
{\bfseries
This chapter mainly deals with advice on the style. This is an extension of the book by a student of the main author's. He describes the various ways of writing style and the fallacies of certain writers nowadays.
}
\section{Project yourself with clarity and develop your own style}
Always write by projecting yourself in the background as the reader. Many writers get bloated by their senses. In this way, your style becomes more reader-friendly.We talked about this in \ref{chap:cr} or specifically \ref{sec:sr}.
\section{Suitable design with profound use of verbs and nouns}
The style of writing should incoporate the use of strong verbs and nouns instead of adjectives and adverbs. Adjectives cannot make a weak noun strong or an adverb cannot do the same for a verb. However, this does not mean we should not use them.Inappropriate adverbs should be avoided.
\section{Revise and avoid overdoing it}
Revision of material is a must. A writing revised three times will definitely be better than a first-time written one since it has already been improved upon. Also, the author suggests avoiding using too fancy and boastful words and instead keeping it simple.
\section{Breezy style}
Avoid writing in a style which is boastful. Readers are easily turned off by people who obsess over themselves in writing and think that the world revolves around them. Instead, use language that is not pretentious but compact and informative.
\section{Use of orthodox spelling}
Basically, your writing should contain the orthodox spellings of words and not the slang versions, such `nite' for night and `thru' for through.
\section{Avoid explaining too much}
Too much explaining means that the writer is not confident about his writing. He believes he/she is not clear. This can be an indication to many readers that the prose will not be up to the mark. Hence, you have dropped the quality of the prose.
\section{Avoid opinion pushing,dialects and foreign langauge}
Pushing opinions on the readers means that your writing will already be derived of certain readers. Rather, use logic and reason to pursue your argument. Also, avoid foreign language or dialect. Even though they sound and look cool, many readers will find uncomfortable in reading them.
\section{Stay to the standards and avoid shortcuts}
In conclusion, the author advises to stay to the standards and void all sorts of shortcuts in the writing. Many classes of language have now dominated the world: business,advertising etc. However, the standard prose system is still the most preferred and easy to read.


\chapter{Conclusion}
In conclusion, I would personally like to say that studying this book has already helped me improve my english a lot. Even though the last edition was updated in 1923, this book is still one of the most profound books on English Language and grammar. Any student, taking an course on English anywhere, should read up on this book.
\\
\\
\style{\stylus{The}}{\stylus{End}}

\end{document}

